\documentclass[12pt]{article}
\usepackage{amsmath, amssymb, amsthm, commath, enumitem, graphicx, nopageno, quoting}
\usepackage[margin=1cm]{geometry}

\title{Computer Science 452 - Homework Assignment \#5}
\author{Hari Amoor, NetID: hra25}
\date{February 18, 2020}

\begin{document}
\maketitle

\section*{Problem 2.11: Supply a PDA for a given CFG $G_{4}$.}
We supply this equivalent CFG $G$ in Greibach-Normal Form:
\begin{align*}
  S \rightarrow E \nonumber \\
  E \rightarrow K \mid T \nonumber \\
  T \rightarrow L \mid M \mid Y \mid Z \nonumber \\
  K \rightarrow E \mid T \nonumber \\
  Y \rightarrow (E) \nonumber \\
  Z \rightarrow a \nonumber
\end{align*}
Now, we generate a PDA $P$ for $G$. $P$ has only one state $q$, and the start symbol of $P$ is the start symbol $S$ of $G$. Furthermore, the input symbols in $P$ are the terminal symbols $a, (, )$, in $G$ and the stack symbols will be the non-terminal symbols iN $G$. The transition rules are as follows:

\begin{align*}
  \delta(q, \epsilon, S) = (q, E) \nonumber \\
  \delta(q, \epsilon, E) = \{(q, K), (q, T)\} \nonumber \\
  \delta(q, ), E) = (q, \epsilon) \nonumber \\
  \delta(q, \epsilon, T) = \{(q, K), (q, T), (q, Y), (q, Z)\} \nonumber \\
  \delta(q, \epsilon, K) = \{(q, E), (q, T)\} \nonumber \\
  \delta(q, (, Y) = (q, E) \nonumber \\
  \delta(q, a, Z) = (q, a) \nonumber
\end{align*}

\section*{Problem 2.20, Claim: If $A$ is context-free and $B$ is regular, then $A \setminus B$ as defined is context-free.}
\begin{proof}
  Let $X$ be a PDA for $A$ and $Y$ a finite automaton for $B$. We provide a PDA $Z$ that recognizes $A \setminus B$. \\
  \newline
  We initialize $Z$ with $Q_{Z} = Q_{X} \times Q_{Y}$. First, $Z$ reads the input word $w$ and advances in $X$, ignoring $Y$. When and if the $Z$ reaches the end of $w$, i.e. by coming to a success state in $X$, $Z$ guesses a word $x$ that will continue to advance in $X$ and start to advance in $Y$; in order for $Z$ to accept $w$, $x$ has to reach an acceptance state in both automata simultaneously. \\
  \newline
  If there is an accepting run of $Z$, then $w \in A \setminus B$ since it guessed a witness $x$. Similarly, if there exists an $x$ satisfying $wx \in A$, then it will be discovered by $Z$. Thus, since $Z$ exists, it must be the case that $A \setminus B$ is context-free.
\end{proof}

\section*{Problem 2.25, Claim: The set of CFLs is closed under the $\text{SUFFIX}$ operator, as defined.}
\begin{proof}
  Let $A$ be a CFL and $H$ be a CFG that generates $A$. We define a CFG $L$ for $\text{SUFFIX}(A)$ as follows: \\
  \newline
  Suppose w.l.o.g. that $H$ is in Chomsky-Normal Form. First, for each variable $X$ in $H$, include variables $X$ and $X'$ in $L$. Now:
  \begin{enumerate}
  \item For every rule in $H$ of the form $X \rightarrow YZ$, include the following rules into $L$:
    \begin{align*}
      X \rightarrow YZ \nonumber \\
      X' \rightarrow Y'Z \mid Z'
    \end{align*}
  \item For every rule in $H$ of the form $X \rightarrow \sigma$, include the following rules into $L$:
    \begin{align*}
      X \rightarrow \sigma \nonumber \\
      X' \rightarrow \sigma \mid \epsilon \nonumber
    \end{align*}
  \end{enumerate}
  The CFL generated by $L$ is exactly equal to $\text{SUFFIX}(A)$, as required.
\end{proof}

\section*{Problem 2.30(a, d): Use the pumping lemma to show that the following languages are not context-free.}
\begin{enumerate}
\item \textbf{The given language $A$.}
  \begin{proof}
    Suppose for contradiction that $A$ is context-free. The string $s = 0^{k}1^{k}0^{k}1^{k}$ is in $A$, where $k$ is the pumping length of $A$. Since $s$ is sufficiently large, the decomposition $s = uvxyz$ must exist as per the pumping lemma. \\
    \newline
    Now, the following cases are exhaustive: \\
    \newline
    \textbf{Case 1:} $vxy$ contains all 0s and is contained within the first or second string of 0s. \\
    \newline
    Here, since $\abs{vy} > 0$, either $v$ or $y$ must contain at least one 0. Now, observe that $uv^{0}xy^{0}z \notin A$; this contradicts the pumping lemma.
    \textbf{Case 2:} $vxy$ contains all 1s and is contained within the first or second string of 1s. \\
    \newline
    By the same reasoning in Case 1, it is clear that a contradiction can be derived. \\
    \newline
    \textbf{Case 3:} $vxy$ contains both 0s and 1s. \\
    \newline
    Suppose w.l.o.g. that $vxy$ contains 0s followed by 1s. Then, $vxy$ straddles either the first division between 0s and 1s or the second division. Now, because $\abs{vxy} \leq k$, it holds that pumping up or down will only affect the substrings immediately adjacent to the division being straddled; the other two substrings would be unaffected. Thus, pumping $s$ will result in a string not in the language, which contradicts the pumping lemma. \\
    \newline
    Consequently, the claim holds.
  \end{proof}
\item \textbf{The given language $L$.}
  \begin{proof}
    Suppose for contradiction that $L$ is context-free, and let $p$ denote its pumping length. Consider $s = 0^{p}1^{p}\#0^{p}1^{p} \in L$. The following cases are exhaustive: \\
    \newline
    \textbf{Case 1:} $vxy$ does not contain the $\# $ symbol. \\
    \newline
    Here, it must be the case that $vxy$ is entirely on one side of the $\#$ symbol; then, pumping $v$ and $y$ will result in a string not in $L$, which is a contradiction. \\
    \newline
    \textbf{Case 2:} Either $v$ or $y$ contains the symbol $\#$ (assume $v$ w.l.o.g.). \\
    \newline
    Here, $uv^{0}xy^{0}z \notin L$ because it does not contain the $\#$ symbol. This contradicts the pumping lemma.
    \textbf{Case 3:} $x$ contains the symbol $\#$. \\
    \newline
    Here, $v$ is a substring of the first string of 1s, and $y$ is a substring of the second string of 0s. Thus, $uv^{0}xy^{0}z \notin L$ because there is either a smaller number of 1s in $t_{1}$ or a larger number of 0s in $t_{2}$. This is, therefore, a contradiction. \\
    \newline
    Thus, the claim holds.
  \end{proof}
\end{enumerate}

\section*{Problem 2.32, Claim: The language $C$, as given, is not context-free.}
\begin{proof}
  Suppose for contradiction that $C$ is context-free with pumping length $p$. Let $s = 1^{p}3^{p}2^{p}4^{p}$. Clearly, the decomposition $s = uvxyz$ must exist.\\
  \newline
  W.l.o.g., suppose $vxy$ includes the symbol 1. Then, it cannot include 2. With this, $uv^{2}xy^{2}z \notin C$, because the number of 1s in the same is not equal to the number of 2s. \\
  \newline
  This directly contradicts the pumping lemma. Thus, the claim holds.
\end{proof}

\section*{Problem 2.46: Show that the grammar $G$, as given is ambiguous, and provide an unambiguous grammar for the same.}
Let $L$ be the language of strings of the form $a^{n}b^{n}$. $L(G)$ is the language $\{x_{1}x_{2} \ldots x_{k} \mid x_{i} \in L\}$. \\
\newline
Consider the string $s = x_{1}x_{2}x_{3}$ where $x_{i} = ab$. Here, $s$ can be parsed either as two separate subexpressions ($x_{1}x_{2}, x_{3}$) or as two separate subexpressions ($x_{1}, x_{2}x_{3}$). Thus, since $s$ has two separate parse trees, the claim holds that this is ambiguous. \\
\newline
An unambiguous CFG $H$ is provided as follows:
\begin{align*}
  S \rightarrow TS \mid T \nonumber \\
  T \rightarrow aTb \mid ab \nonumber
\end{align*}
In order to prove that $H$ is unambigious, we let $s \in \Sigma^{\star}$ be arbitrary, and show that $s$ has only one parse tree under $H$.

\section*{Problem 2.48: Prove the following.}
\begin{enumerate}[label=(\alph*)]
\item \textbf{Claim: $C_{1}$, as given, is a CFL.}
  \begin{proof}
    The following CFG generates $C_{1}$:
    \begin{align*}
      S \rightarrow T1T \nonumber \\
      T \rightarrow 0T \mid 1T \mid 0 \mid 1 \nonumber
    \end{align*}
    Thus, the claim holds that $C_{1}$ is context-free.
  \end{proof}
\item \textbf{Claim: $C_{2}$ is not context-free.}
  \begin{proof}
    Suppose for contradiction that $C_{2}$ is a CFL, and let $p$ be its pumping length. Let $s = w10^{p}w$, where $w = 0^{p+2}$. Then, the decomposition $s = uvxyz$ is well-defined w.r.t. the pumping lemma. \\
    \newline
    The following cases are exhaustive: \\
    \newline
    \textbf{Case 1:} $vxy$ contains one, but not both, of the 1s in $s$. \\
    \newline
    Here, if the 1 is in $x$, we are done; with $v$ and $y$ pumped to a sufficiently large length, we can generate a string not in $C_{2}$. So, suppose w.l.o.g. that the 1 is in $v$. Then, clearly $uv^{2}xy^{2}z \notin C_{2}$, because you can pump at most two of the strings of 0s. This contradicts the pumping lemma. \\
    \newline
    \textbf{Case 2:} $vxy$ does not contain either of the 1s in $s$. \\
    \newline
    Pumping $v$ and $y$ here will result in arbitrarily many 0s; thus, when done to a sufficiently large length, one can generate a string outside of $C_{2}$, which is a contradiction. \\
    \newline
    Thus, the claim holds.
  \end{proof}
\end{enumerate}

\end{document}
