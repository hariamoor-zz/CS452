\documentclass[12pt]{article}
\usepackage{amsmath, amssymb, amsthm, commath, enumitem, quoting}
\usepackage[margin=1cm]{geometry}

\title{Math 351 - Homework Assignment \#1}
\author{Hari Amoor, NetID: hra25}
\date{September 16, 2019}

\begin{document}
\maketitle

\section*{Problem 0.7: Supply a relation that satisfies each constraint.}
\begin{enumerate}[label=(\alph*)]
	\item \textbf{Reflexive and symmetric, but not transitive.} \\
		\newline
		The relation between two people $x, y$ given iff $x$ is blood-related to $y$ is reflexive and symmetric, but not transitive.
	\item \textbf{Reflexive and transitive, but not symmetric.} \\
		\newline
		$R = \{(a, b) \in \mathbb{Z} \times \mathbb{Z} : a \geq b\}$ is reflexive and transitive, but not symmetric.
	\item \textbf{Symmetric and transitive, but not reflexive.} \\
		\newline
		$R = \{(0, 0), (0, 1), (1, 0), (1, 1)\}$ in $X = \{0, 1, 2\}$. This is symmetric and transitive, but not reflexive since $(2, 2) \notin R$.
\end{enumerate}

\section*{Problem 0.9: Formally describe the given graph.}
The given graph is the complete bipartite graph $K_{3, 3}$ with formal description $(\{1, 2, 3, 4, 5, 6\}, \{(1, 4), (2, 4), (3, 4), (1, 5), (2, 5), (3, 5), (1, 6), (2, 6), (3, 6)\}$.

\section*{Problem 0.10: Find the error in the given proof.}
The error in the given proof is that the writer defines $a = b = 1$. Thus, when he transforms the equality $(a + b)(a - b) = b(a + b)$ to $a + b = b$, he divides by $a - b = 0$. Thus, the proof is incorrect.

\section*{Problem 0.12: Find the error in the given proof.}
Let $P(h)$ be the predicate that all horses in a set of $h$ horses are the same color. As per the base case, $P(1)$ is true. However, $P(2)$ is not implied by the base case $P(1)$. \\
\newline
Take two horses $x, y$ of different color. By the inductive hypothesis, all the horses in $\{x\}$ and those in $\{y\}$ are the same color. However, the horses in $\{x, y\}$ are not the same color by definition. Thus, the claim of the author that $P(k)$ implies $P(k + 1)$ is untrue.

\section*{Problem 0.13, Claim: It is not true that every graph of two or more vertices contain two vertices of equal vertices when self-loops are allowed.}
\begin{proof}
The graph $G$ with two vertices $x, y$ where $x$ has a self-loop and $y$ has no edge is a counterexample.
\end{proof}

\section*{Problem 0.14, Claim: Ramsey's Theorem.}
\begin{proof}
	Let $G$ be a graph of size $n$. Now, assume a 2-coloring on the edges of the complete graph $K_{n}$ of size $n$ s.t. red edges are in $G$ and blue edges are not in $G$. \\
	\newline
	If there are $k \geq \frac{1}{2}\lg{n}$ red edges, then we are done. So, assume that this is not the case, i.e. $k \le \frac{1}{2}\lg{n}$. Therefore, there are $n^{2} - n - k$ blue edges in our coloring of $K_{n}$. \\
	\newline
	Finally, observe that $k + \frac{1}{2}\lg{n} \le \lg{n} \leq n(n - 1) = n^{2} - n$. Thus, $n^{2} - n - k \ge \frac{1}{2}\lg{n}$; so, it follows that there are more than $\frac{1}{2}\lg{n}$ blue edges. \\
	\newline
	We know that the blue edges of $K_{n}$ are not connected by an edge in $G$. Thus, the set of vertices connected with blue edges in $K_{n}$ forms an independent set in $G$, as required.
\end{proof}

\section*{Extra Problem, Claim: The relation $S$ defined by $(1, 1) \in S$ and $(a + 1, b + 2a + 1) \in S$ for all $(a, b) \in S$ is equivalent to the function $f: \mathbb{N} \rightarrow \mathbb{N}$ induced by $n \mapsto n^{2}$.}
\begin{proof}
	Let $P(n)$ be the predicate that $(n, n^{2}) in S$ for a particular $n$. We prove the claim with mathematical induction. \\
	\newline
	\underline{Base Case:} It is given that $(1, 1) = (1, 1^{2}) \in S$. Thus, $P(1)$ holds. \\
	\newline
	\underline{Inductive Step:} Suppose that $P(k)$ holds for all $k$ s.t. $1 \leq k \leq m$, where $m \geq 1$. It follows from the inductive hypothesis that $m + 1 \mapsto m^{2} + 2m + 1 = (m + 1)^{2}$. Thus, $P(m + 1)$ holds. \\
	\newline
	Hence, by the Principle of Mathematical Induction, the claim holds.
\end{proof}

\end{document}
